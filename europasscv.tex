\documentclass[italian,a4paper]{europasscv}
\overfullrule=2cm
\ecvname{Enrico Tolotto}
\ecvaddress{Via Vanzo N1\newline Motta di Livenza, 31045 TV.}
\ecvtelephone[+39 3400665982]{+39 0422860359}
\ecvemail{etolotto@gmail.com}
\ecvpicture[width=3.5cm]{Image.eps}
\ecvdateofbirth{17 Novembre 1994}
\ecvnationality{Italiana}
\ecvgender{Maschio}

% \ecvpicture[width=3.8cm]{picture.jpg}

\begin{document}
	\begin{europasscv}
	\ecvpersonalinfo
	\ecvbigitem{Ocupazione desiderata} {Ingegnere elettronico}
	\ecvsection{Istruzione e formazione}
	\ecvtitle{2013 -- 2017} {
		Dottore in Ingegneria Elettronica
	}
	\ecvitem{} {
		Università degli studi di Udine
	}
	\ecvitem{} {
		\textcolor{ecvhighlightcolor}{Titolo Tesi}:
	}
	\begin{center} {
		\setstretch{1.5}
		LoRa e Internet Of Things:\linebreak
		Everyware Software Come Caso di Studio.
	}
	\end{center}
	\ecvitem{} {
		\begin{ecvitemize}
			\item Telecomunicazioni.
			\item Embedded Systems.
			\item Analisi e progettazione del Software.
			\item Progettazione sistemi analogici e digitali.
			\item Circuiti ad alte frequenze.
		\end{ecvitemize}
	}

	\ecvtitle{Settembre 2015 -- Giugno 2016} {
		Erasmus
	}
	\ecvitem{} {
		Sabanci University Turchia
	}
	\ecvitem{} {
		\begin{ecvitemize}
			\item Introduction to probability.
			\item Magnetic fields.
			\item Einstein relativity.
			\item Basic turkish.
			\item Microwaves.
			\item Electronic circuits.
		\end{ecvitemize}
	}

	\ecvtitle{2008 -- 2013} {
		Perito industriale meccanico
	}
	\ecvitem{} {
		Istituto Tecnico Superiore A.Scarpa\newline Motta di Livenza
	}
	\ecvitem{} {
		\begin{ecvitemize}
			\item Disegno industriale.
			\item Macchine a fluido.
			\item Tecnologia meccanica.
			\item Sistemi e automazione industriale.
		\end{ecvitemize}
	}

	\ecvsection{Esperienze lavorative}

	\ecvtitle{Marzo 2017 -- Ottobre 2017} {
		Tirocinio per Tesi
	}
	\ecvitem{} {
		Eurotech (Italy) \newline 33020 Amaro-UD,
		Via Fratelli Solari, 3/a \newline +39 0433.485.411
	}
	\ecvitem{} {
		\begin{ecvitemize}
				\item Progettazione ed implementazione
						della piattaforma IoT LoRa all'interno del ecosistema ESF.
				\item Utilizzo della suite di sviluppo Kura basata sul framework OSGi.
				\item Scrittura di piccole utility di sistema nel linguaggio Go.
				\item Utilizzo del protocollo Mqtt.
		\end{ecvitemize}
	}

	\ecvtitle{Gennaio 2012 -- Febbraio 2012} {
		Stagista
	}
	\ecvitem{} {
		Dema engineering (Italy)\newline 31046 Oderzo-TV Via Garibaldi, 145
		\newline +39 0422 824040
	}
	\ecvitem{} {
		\begin{ecvitemize}
			\item Progettazione impianti di raffreddamento stampaggio camera calda.
			\item Progettazione di maschere per stampaggio a camera calda di
				casseri e fanali automobili.
			\item Corso di disegno 3D con il software Catia.
		\end{ecvitemize}
	}

	\ecvtitle{Gennaio 2011 -- Febbraio 2011}{
		Stagista
	}
	\ecvitem{} {
		Union Glass (Italy)\newline 31045 Motta di Livenza TV, Via Istria
		\newline +39 0422 861235
	}

	\ecvitem{} {
		\begin{ecvitemize}
			\item Sviluppo di impianti fotovoltaici.
			\item Progettazione di vetri protettivi per impianti fotovoltaici.
		\end{ecvitemize}
	}

	\ecvsection{Competenze personali}

	\ecvblueitem{Lingua madre} {
		Italiana.
	}

	\ecvlanguageheader
	\ecvlastlanguage{Inglese}{C1}{C1}{C1}{C1}{C1}
	\ecvlanguage{Francese}{A1}{A1}{A1}{A1}{A1}
	\ecvlastlanguage{Turco}{A2}{A2}{A2}{A2}{A2}
	\ecvlanguagefooter

	\ecvblueitem{Competenze comunicative} {
		\begin{ecvitemize}
			\item Durante il mio Erasmus ho incontrato molte persone nuove con
				differenti background, questa esperienza ha migliorato le mie
				capacità di comunicazione aiutandomi a superare le mie paure.
				Inoltre la maggior parte delle persone, in Turchia, non parla la
				lingua inglese, facendo si che il mio livello di conoscenza della
				lingua turca sia cresciuto notevolmente nell'arco dei nove mesi
				di soggiorno.
		\end{ecvitemize}
	}

	\ecvblueitem{Competenze organizzative e gestionali} {
	\begin{ecvitemize}
		\item Nel corso dei miei studi di perito meccanico ho condotto, insieme
			ad un altro compagno e un professore, un progetto che ha portato alla
			realizzazione di un pulsoreattore. Durante questo progetto abbiamo
			dovuto organizzare le fasi di progettazione, ingegnerizzazione,
			produzione e test del modello
		\item Durante il mio Erasmus ho dovuto partecipare ed organizzare eventi
			culturali, grazie ai quali ho migliorato le mie competenze di
			collaborazione , programmazione e gestione del lavoro.
	\end{ecvitemize}
	}

	\ecvblueitem{Competenze professionali} {
	\begin{ecvitemize}
		\item Durante il mio futuro percorso lavorativo vorrei approfondire la mia
			conoscenza e competenza in scrittura di driver e programmazione a basso
			livello per dispositivi riguardanti la domotica e IoT, con interesse
			riguardo a progettazione e sviluppo di apparecchi basati su nuovi standard
			comunicativi come LWPA.
	\end{ecvitemize}
	}

	\ecvblueitem{Competenza digitale} {
		\begin{ecvitemize}
			\item buona padronanza degli strumenti del pacchetto Office (word, excel,
				powerpoint)
			\item conoscenze avanzate di linux e sistemi operativi *nix like
			\item conoscenze dei principali linguaggi di programmazione come VHDL, C,
				C++, Java, Python, Matlab, Shell scripting.
			\item conoscenze in programmazione di Arduino o sistemi immersi come
				Raspberry Pi.
			\item buona padronanza dei programmi di disegno tecnico e
				modellazione solida, acquisita durante gli studi usando
				spesso AutoCAD , SolidWorks, Catia.
			\item Strumenti di progettazione come Orcad, ADS, EagelCad.
			\item Patentino ECDL.
		\end{ecvitemize}
	}

	\ecvblueitem{Altre competenze e interessi} {
		Oltre a fare sport sono un appassionato di DIY e del movimento dei
		makers, attivo in svariati blog mi diverto a crea e sperimentare con
		l'elettronica e la programmazione.
	}

	\ecvblueitem{Patente di guida} {
		B ottenuta il 02/02/2013
	}
	\end{europasscv}
\end{document}
